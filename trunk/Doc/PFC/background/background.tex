%=============================================================================%
% Author : Angel Tezanos Iba�ez                                               %
% Author : Pablo S�nchez Barreiro                                             %  % Version: 2.0, 23/02/2011                                                    %
% Master Thesis: Introduction                                                 %
%=============================================================================%

\chapterheader{Antecedentes}{Antecedentes}
\label{chap:introduction}

El presente cap�tulo describe brevemente las tecnolog�as sobre las que se fundamenta el presente proyecto. M�s concretamente, se explica el funcionamiento de ...

\chaptertoc

\section{Desarrollo de Software basado en Componentes}

El proyecto se desarrollara bajo una programaci�n orientada a componentes. Esta rama de la ingenier�a software trata de construir sistemas a base de componentes funcionales, como si de un lego se tratase. Para ello cada componente debe tener una interfaz bien definida.

El nivel de abstracci�n de de los componentes se considera mas alto que el de los objetos al agrupar unidades funcionales autonomamente. De esta manera se explota en gran medida las posibilidades de reutilizaci�n. Pudiendo utilizar componentes ya creados por otros, y/o en otros proyectos de manera r�pida y sencilla.

Cada componente software es un elemento o pieza del sistema final que ofrece un servicio y es capaz de comunicarse con el resto de componentes. b�sicamente un componente es un objeto escrito siguiendo unas especificaciones, si las cumple adquiere la caracter�stica de reusabilidad.

Los componentes deben poder ser serializados para garantizar el envi� del estado del objeto a trav�s de flujos de datos.

Para que un componente este bien dise�ado requiere un esfuerzo en la fase de dise�o, pues se debe tener en cuenta que puede ser reutilizado por muchos programas, debe estar debidamente documentado, probado de manera enf�tica, es decir, se debe probar la validez de las entradas y que sea capaz de mostrar mensajes de error claros y oportunos; tambi�n se debe prever el uso del componente de manera imprevista o incorrecta.

\section{Java beans como modelo de componentes}

JavaBeans es la tecnolog�a de componentes de Java, cada componente se le conoce como bean, como se dijo anteriormente, un bean no es mas que una clase de objetos con unas caracter�sticas especiales:

\begin{enumerate}
	\item Es una clase publica que implementa la interfaz serializable
	\item Expone una serie de propiedades que pueden ser le�das y modificadas por el entorno de desarrollo.
	\item Los evento que posea pueden ser capturados y asociados a una serie de acciones.
\end{enumerate}

% Sin n�mero \begin{itemize}

Las propiedades no son mas que atributos del objeto que pueden ser modificados y le�dos por el entorno de desarrollo. Cada propiedad debe tener al menos un m�todo get para obtener el valor, y un set para modificarlo. En caso de que no se implemente el m�todo set se entender� que es una propiedad de solo lectura.

Existen varios tipos de propiedades:
\begin{itemize}
    \item Simples: Representa un �nico valor
    \item Indexadas: Representa un array de valores
    \item Ligadas (Bound): Notifican un cambio de la propiedad a otros objetos (listeners).
    \item Restringidas (Constrained): Similar a la Ligada salvo que los objetos notificados tienen la opci�n de vetar el cambio.
\end{itemize}

%%Extender mas si es posible