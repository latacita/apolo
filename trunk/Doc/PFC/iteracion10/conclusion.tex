%============================================================================%
% Author : Angel Tezanos Iba�ez                                              %
% Author : Pablo S�nchez Barreiro                                            %
% Version: 2.0, 07/04/2011                                                   %
% Master Thesis: Conclusiones y Trabajos Futuros                             %
%============================================================================%

\chapterheader{Conclusiones y Trabajos Futuros}{Conclusiones y Trabajos Futuros}
\label{chap:discusion}

\todo{P�rrafo introductorio}

\chaptertoc

\section{Conclusiones}
\label{sec:discusion:conclusiones}
    En este proyecto de fin de carrera se ha implementado una aplicaci�n denominada Apolo, la cual es capaz de organizar y clasificar fotograf�as de manera sencilla, r�pida y con una interfaz de usuario amigable. El resto de acciones que permite la aplicaci�n (visualizar y editar fotograf�as) se delegan en los programas que por defecto tenga instalado el sistema operativo para tal fin. Gracias a ello hemos logrado una aplicaci�n altamente �til y con un consumo de recursos bajo en comparaci�n con otros clasificadores de fotograf�as, que sin duda alguna resultan mucho mas \emph{pesados}.

    La aplicaci�n permite importar fotograf�as, y moverlas una a una, hacia una serie de baldas, en las cuales sera depositadas y ordenadas seg�n el orden que desee el usuario. Una vez ordenado un subconjunto de ellas, podr�n ser empaquetadas en subsecuencias de manera que estas subsecuencias tambi�n pueden reordenarse entre ellas.

    Una vez se desee exportar el conjunto de subsecuencias, el usuario podr� elegir entre un nombre com�n para todas las fotograf�as mas una numeraci�n, o solo una numeraci�n. El propio programa se encargada de mostrar la numeraci�n correspondiente, utilizando tantos ceros como sean necesarios, para garantizar una ordenaci�n correcta en cualquier sistema.
    
    El programa tambi�n permite guardar el estado de la aplicaci�n en un fichero, de manera que mas adelantes sea capaz de recuperable y continuar con el trabajo. Al guardar el estado, deberemos elegir el nombre del fichero y el directorio donde deseamos que se guarde. Tambi�n nos ofrece la posibilidad de elegir que deseamos guardar: La zona superior donde se almacenan los subconjuntos de diapositivas ordenadas; la zona central, donde se encuentran la estanter�a con sus baldas; o la zona inferior o mesa, con todo el pool de diapositivas.

\section{Trabajos Futuros}
\label{sec:discusion:trabajosFuturos} 
    En el futuro mas proximo se creara instaldores para los sistemas Mac, de manera que la aplicaci�n sea f�cil de instalar en el pr�cticamente 100\% de los sistemas operativos de escritorio. 
    
    En un futuro mas lejano y si la aplicaci�n tiene aceptaci�n por parte de los usuarios, esta pensado crear un modulo que permita a la aplicaci�n, exportar las fotograf�as a trav�s de la red. Es decir, la funcionalidad buscada es que un usuario con un conjunto de subsecuencias  ordenadas, pueda en un momento dado dejar un socket abierto en el sistema a la espera de peticiones, y que otro usuario con Apolo se pueda conectar a dicho socket y se intercambien fotograf�as. De esta manera seria posible intercambiar fotograf�as sin tener que esperar a ver a nuestro amigo con las fotograf�as y nosotros con el USB a mano. 