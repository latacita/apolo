%=============================================================================%
% Author : Angel Tezanos Iba�ez                                               %
% Author : Pablo S�nchez Barreiro                                             %  % Version: 2.0, 23/02/2011                                                    %
% Master Thesis: Introduction                                                 %
%=============================================================================%

\chapterheader{Antecedentes}{Antecedentes}
\label{chap:introduction}

El presente cap�tulo describe brevemente las tecnolog�as sobre las que se fundamenta el presente proyecto. M�s concretamente, se explica el funcionamiento de ...

\chaptertoc

\section{Desarrollo de Software basado en Componentes}

El proyecto se desarrollara bajo una programacion orientada a componentes. Esta rama de la ingenier�a software trata de construir sistemas a base de componentes funcionales, como si de un lego se tratase. Para ello cada componente debe tener una interfaz bien definida.

El nivel de abstracci�n de de los componentes se considera mas alto que el de los objetos al agrupar unidades funcionales autonomamente. De esta manera se explota en gran medida las posibilidades de reutilizacion. Pudiendo utilizar componentes ya creados por otros, y/o en otros proyectos de manera rapida y sencilla.

Cada componente software es un elemento o pieza del sistema final que ofrece un servicio y es capaz de comunicarse con el resto de componentes. basicamente un componente es un objeto escrito siguiento unas especificaciones, si las cumple adquiere la caracteristica de reusabilidad.

Los componentes deben poder ser serializados para garantizar el envio del estado del objeto a traves de flujos de datos.

Para que un componente este bien dise�ado requiere un esfuerzo en la fase de dise�o, pues se debe tener en cuenta que puede ser reutilizado por muchos programas, debe estar debidamente documentado, probado de manera enfatica, es decir, se seben probar la validez de las entradas y que sea capaz de mostrar mensajes de error claros y oportunos; tambien se debe preveer el uso del componente de manera imprevista o incorrecta.

\section{Java beans como modelo de componentes}

JavaBeans es la tecnologia de componentes de Java, cada componente se le conoce como bean, como se dijo anteriormente, un bean no es mas que una clase de objetos con unas caracteristicas especiales:

\begin{enumerate}
	\item Es una clase publica que implementa la interfaz serializable
	\item Expone una seria de propiedades que pueden ser leidas y modificadas por el entorno de desarrollo.
	\item Los evento que posea pueden ser capturados y asociados a una serie de acciones.
\end{enumerate}

% Sin n�mero \begin{itemize}

Las propiedades no son mas que atributos del objeto que pueden ser modificados y leidos por el entorno de desarrollo. Cada propiedad debe tener al menos un metodo get para obtener el valor, y un set para modificarlo. En caso de que no se implemente el metodo set se entendera que es una propiedad de solo lectura.