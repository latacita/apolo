\cdpchapter{Resumen}

El objetivo del presente proyecto fin de carrera para acceder al titulo de Ingenier�a Inform�tica de la Universidad de Cantabria,  es crear una aplicaci�n software sencilla, usable y ligera para la clasificaci�n de un conjunto de im�genes digitales. \newline

La interfaz gr�fica de dicha aplicaci�n deber� ser lo m�s parecida posible a un clasificador retroiluminado de diapositivas anal�gicas. El usuario deber� interactuar con la aplicaci�n como si un clasificador de este tipo se tratase. Una vez seleccionadas clasificadas y ordenadas las fotograf�as de una colecci�n de im�genes, la aplicaci�n debe permitir exportarlas a una carpeta concreta como un conjunto de archivos, ordenadas de manera adecuada.
\newline

Las aplicaciones existentes actualmente no tienen como finalidad principal la ordenaci�n y clasificaci�n de fotograf�as, por lo que, aunque muchas de ellas son capaces de clasificar y ordenar fotograf�as, no permiten renombrar las im�genes de manera efectiva, y para su ejecuci�n requieren gran cantidad de recursos del sistema.
\newline

La finalidad del proyecto es crear un software capaz de clasificar y organizar fotograf�as con un consumo de recursos bajo y una curva de aprendizaje muy suave.
\newline

El software ser� desarrollado en Java, utilizando un desarrollo basado en componentes, adoptando una metodolog�a iterativa e incremental.


\cdpchapter{Preface}

The purpose of this Thesis Project, part of the Computer Engineering degree from the
University of Cantabria, is to create a simple software application, usable for the classification of a set of digital images.
\newline

The Graphic User Interface of the application should be as close as possible to a analog backlit slide sorter. The user must interact with the application as a classifier of that kind. Once photographs from a collection are classified, ordered and selected, the application should be able to export to a particular folder as a set of files, sorted properly.
\newline

Existing applications do not currently have as main purpose the management photos, many of them are able to classify and sort pictures, but you can't rename images in an effective way, and usually require significant system's resources to run properly.
\newline

The project aims to create a software that can sort and organize photos with a low usage of system's resources and a very soft learning curve.
\newline

The software will be developed in Java using a component-based development, adopting an iterative and incremental approach. 