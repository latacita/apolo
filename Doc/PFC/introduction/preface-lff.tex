\cdpchapter{Resumen}

El objetivo del presente Proyecto Fin de Carrera es crear una aplicaci�n software sencilla, usable y ligera para la clasificaci�n de un conjunto de im�genes digitales.


La interfaz gr�fica de dicha aplicaci�n debe ser lo m�s parecida posible a un clasificador retroiluminado de diapositivas anal�gicas. El usuario debe interactuar con la aplicaci�n como si un clasificador de este tipo se tratase. Una vez seleccionadas clasificadas y ordenadas las fotograf�as de una colecci�n de im�genes, la aplicaci�n debe permitir exportarlas a una carpeta concreta como un conjunto de archivos, ordenadas de manera adecuada.

Las aplicaciones existentes actualmente no tienen como finalidad principal la ordenaci�n y clasificaci�n de fotograf�as, por lo que, aunque muchas de ellas son capaces de clasificar y ordenar fotograf�as, no permiten renombrar las im�genes de manera efectiva. Adem�s, para su ejecuci�n requieren gran cantidad de recursos del sistema.

La finalidad del proyecto es crear un software capaz de clasificar y organizar fotograf�as con un consumo de recursos bajo y una curva de aprendizaje muy suave. El software ser� desarrollado en Java, utilizando un desarrollo basado en componentes, adoptando una metodolog�a iterativa e incremental.

\paragraph{Palabras Clave} \ \\

Im�genes, Fotograf�as, Clasificaci�n de diapositivas, JavaBeans, Drag{\&}Drop

\cdpchapter{Preface}

This Master Thesis aims to create a simple software application, with a high usability degree, for the ordering classification of a set of digital images.

The application user interface should be as similar as possible to the traditional backlight slide sorters commonly used for the selection, classification and ordering of analogic slides. The user must interact with the application such as if he or she were using one of these classical slide sorters. Once a set photographs from a collection has been properly classified, ordered and selected, the application will have to be able to export these images to a folder specified by the user, ensuring that the ordering between these images is kept.

Similar state-of-art applications do not have as a main goal keeping the ordering between images once they are exported. Most of these applications are able to classify and sort pictures, but the ordering is lost when the images are displayed outside the application. Moreover, these application are often too much heavyweight, demanding a high amount of computer resources.

Thus, the project aims to create a lightweight and usable software application that can sort and organise photos, and whose learning curve is very small. The software will be developed in Java using a component-based development approach. An iterative and incremental software development process is followed. 

\paragraph{Keywords} \ \\

Images, Photography, Pictures, SlideClassification, JavaBeans, Drag{\&}Drop
